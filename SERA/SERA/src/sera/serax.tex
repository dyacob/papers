\section{Ethiopic Letters}

The Ethiopic writing system is a rich syllabary of at least 41
consonant classes each having generally 7 or 8 forms, and fewer
having 12 or 13.  The script has built in mechanisms for the 
extension to additional consonant classes -an occasional need
as members of Ethiopia's and Eritrea's more than 70 language groups
reach greater maturity in their own writing practices. \\

Cataloging of the complete writing system remains an on going
task of the Academy of Ethiopian Languages (a division of Ethiopia's
Ministry of Culture) in 1996 Ethiopia.  Needless to say then that
the Unicode and ISO-10646 introduction of Ethiopic in 1996 will
not be comprehensive.  Indeed, the character code assignment of nearly
50 Ethiopic letters will be left for future extensions to
Unicode/ISO-10646. \\
The SERA system preserves the flexibility of the 
Ethiopic syllabary to adapt to changing needs.  
\newpage
\include{fidelmap3}
\noi
SERA offers sound conventions suitable for any of 
the world's syllabic writing systems.  The Ethiopic 
syllabary will be more readily know to its users by \\
%\onecolumn

\noi
\parbox{180mm}{
{\large\bf The Ethiopic Script in ASCII} \\

%\hspace*{-0.15in}
%\hspace*{-0.7in}
%\subsection*{The SERA Fidel Syllabary}
\noi
%\hspace*{-0.7in}
\begin{tabular}{|*{14}{c|}} \hline
     &  1 &   2  &  3 &  4 &  5 &  6 &  7 &  8 & (12) & 9 & 10 & 11 & 12  
\\ \hline
     & \g\II\z\ & \ka\II\b\ & \sa\l\s\ & \ra\b\II\ & \ha\m\s\ & \sa\d\s\ & \sa\b\II\ & \multicolumn{6}{|l|}{\di\qa\la \ $\rightarrow$}
\\ \hline
 \he & {\em he} & {\em hu} & {\em hi}  & {\em ha}  & {\em hE}  & {\em h}  & {\em ho} &          &          &           &           &           &
\\ \hline
 \le & {\em le} & {\em lu} & {\em li}  & {\em la}  & {\em lE}  & {\em l}  & {\em lo} &          &          &           & {\em lWa} &           &
\\ \hline
 \He & {\em He} & {\em Hu} & {\em Hi}  & {\em Ha}  & {\em HE}  & {\em H}  & {\em Ho} &          &          &           & {\em HWa} &           &
\\ \hline
 \me & {\em me} & {\em mu} & {\em mi}  & {\em ma}  & {\em mE}  & {\em m}  & {\em mo} & {\em mWe} & ({\em mWu}) & {\em mWi}  & {\em mWa}  & {\em mWE}  & {\em mW} 
\\ \hline
 \sse & {\em `se} & {\em `su} & {\em `si}  & {\em `sa}  & {\em `sE}  & {\em `s}  & {\em `so}&   &          &           & {\em `sWa}&           &
\\ \hline
 \re & {\em re} & {\em ru} & {\em ri}  & {\em ra}  & {\em rE}  & {\em r}  & {\em ro} &          &          &           & {\em rWa} &           &
\\ \hline
 \se & {\em se} & {\em su} & {\em si}  & {\em sa}  & {\em sE}  & {\em s}  & {\em so} &          &          &           & {\em sWa} &           &
\\ \hline
 \xe & {\em xe} & {\em xu} & {\em xi}  & {\em xa}  & {\em xE}  & {\em x}  & {\em xo} &          &          &           & {\em xWa} &           &
\\ \hline
 \qe & {\em qe}  & {\em qu} & {\em qi}  & {\em qa}  & {\em qE}  & {\em q}  & {\em qo} & {\em qWe} & ({\em qWu}) & {\em qWi}  & {\em qWa}  & {\em qWE}  & {\em qW}
\\ \hline
 \qqe & {\em `qe}  & {\em `qu} & {\em `qi}  & {\em `qa}  & {\em `qE}  & {\em `q}  & {\em `qo} &  &  &  &   &  & 
\\ \hline
 \Qe & {\em Qe} & {\em Qu} & {\em Qi}  & {\em Qa}  & {\em QE}  & {\em Q}  & {\em Qo} & {\em QWe} & ({\em QWu}) & {\em QWi}  & {\em QWa}  & {\em QWE}  & {\em QW}  
\\ \hline
 \be & {\em be} & {\em bu} & {\em bi}  & {\em ba}  & {\em bE}  & {\em b}  & {\em bo} & {\em bWe}&({\em bWu})& {\em bWi} & {\em bWa} & {\em bWE} & {\em bW}
\\ \hline
 \ve & {\em ve} & {\em vu} & {\em vi}  & {\em va}  & {\em vE}  & {\em v}  & {\em vo} &          &          &           & {\em vWa} &           &
\\ \hline
 \te & {\em te} & {\em tu} & {\em ti}  & {\em ta}  & {\em tE}  & {\em t}  & {\em to} &          &          &           & {\em tWa} &           &
\\ \hline
 \ce & {\em ce} & {\em cu} & {\em ci}  & {\em ca}  & {\em cE}  & {\em c}  & {\em co} &          &          &           & {\em cWa} &           &
\\ \hline
 \hhe & {\em `he} & {\em `hu} & {\em `hi}  & {\em `ha}  & {\em `hE}  & {\em `h}  & {\em `ho} & {\em hWe} &({\em hWu})& {\em hWi}  & {\em hWa}  & {\em hWE}  & {\em hW} 
\\ \hline
 \ne & {\em ne} & {\em nu} & {\em ni}  & {\em na}  & {\em nE}  & {\em n}  & {\em no} &          &          &           & {\em nWa} &           &
\\ \hline
 \Ne & {\em Ne} & {\em Nu} & {\em Ni}  & {\em Na}  & {\em NE}  & {\em N}  & {\em No} &          &          &           & {\em NWa} &           &
\\ \hline
 \ea & {\em e/a}$^{\ast}$ & {\em u/U} & {\em i}  & {\em A/a}  & {\em E}  & {\em I}  & {\em o/O} & {\em ea} & & & & & 
\\ \hline
 \ke & {\em ke} & {\em ku} & {\em ki}  & {\em ka}  & {\em kE}  & {\em k}  & {\em ko} &{\em kWe} & ({\em kWu}) & {\em kWi}  & {\em kWa}  & {\em kWE}  & {\em kW}
\\ \hline
 \kke & {\em `ke}  & {\em `ku} & {\em `ki}  & {\em `ka}  & {\em `kE}  & {\em `k}  & {\em `ko} &  &  &  &   &  & 
\\ \hline
 \Ke & {\em Ke} & {\em Ku} & {\em Ki}  & {\em Ka}  & {\em KE}  & {\em K}  & {\em Ko} & {\em KWe} & ({\em KWu}) & {\em KWi}  & {\em KWa}  & {\em KWE}  & {\em KW}
\\ \hline
 \Xe & {\em Xe}  & {\em Xu} & {\em Xi}  & {\em Xa}  & {\em XE}  & {\em X}  & {\em Xo} &  &  &  &   &  & 
\\ \hline
 \we & {\em we} & {\em wu} & {\em wi}  & {\em wa}  & {\em wE}  & {\em w}  & {\em wo} &          &          &           &           &           &
\\ \hline
 \ee & {\em `e} & {\em `u/`U} & {\em `i}  & {\em `A/`a}  & {\em `E}  & {\em `I}  & {\em `o/`O} &      &       &      &       &      &
\\ \hline
 \ze & {\em ze} & {\em zu} & {\em zi}  & {\em za}  & {\em zE}  & {\em z}  & {\em zo} &          &          &           & {\em zWa} &           &
\\ \hline
 \Ze & {\em Ze} & {\em Zu} & {\em Zi}  & {\em Za}  & {\em ZE}  & {\em Z}  & {\em Zo} &          &          &           & {\em ZWa} &           &
\\ \hline
 \ye & {\em ye} & {\em yu} & {\em yi}  & {\em ya}  & {\em yE}  & {\em y}  & {\em yo} &          &          &           & {\em yWa} &           &
\\ \hline
 \de & {\em de} & {\em du} & {\em di}  & {\em da}  & {\em dE}  & {\em d}  & {\em do} &          &          &           & {\em dWa} &           &
\\ \hline
 \De & {\em De}  & {\em Du} & {\em Di}  & {\em Da}  & {\em DE}  & {\em D}  & {\em Do} &  &  &  & {\em DWa}  &    &
\\ \hline
 \je & {\em je} & {\em ju} & {\em ji}  & {\em ja}  & {\em jE}  & {\em j}  & {\em jo} &          &          &           & {\em jWa} &           &
\\ \hline
 \ge & {\em ge} & {\em gu} & {\em gi}  & {\em ga}  & {\em gE}  & {\em g}  & {\em go} & {\em gWe}& ({\em gWu}) & {\em gWi}  & {\em gWa}  & {\em gWE}  & {\em gW}
\\ \hline
 \gge & {\em `ge}  & {\em `gu} & {\em `gi}  & {\em `ga}  & {\em `gE}  & {\em `g}  & {\em `go} &  &  &  &   &  & 
\\ \hline
 \Ge & {\em Ge}  & {\em Gu} & {\em Gi}  & {\em Ga}  & {\em GE}  & {\em G}  & {\em Go} & {\em GWe} & ({\em GWu}) & {\em GWi}  & {\em GWa}  & {\em GWE}  & {\em GW} 
\\ \hline
 \Te & {\em Te} & {\em Tu} & {\em Ti}  & {\em Ta}  & {\em TE}  & {\em T}  & {\em To} &          &          &           & {\em TWa} &           &
\\ \hline
 \Ce & {\em Ce} & {\em Cu} & {\em Ci}  & {\em Ca}  & {\em CE}  & {\em C}  & {\em Co} &          &          &           & {\em CWa} &           &
\\ \hline
 \Pe & {\em Pe} & {\em Pu} & {\em Pi}  & {\em Pa}  & {\em PE}  & {\em P}  & {\em Po} &          &          &           & {\em PWa} &           &
\\ \hline
 \Se & {\em Se} & {\em Su} & {\em Si}  & {\em Sa}  & {\em SE}  & {\em S}  & {\em So} &          &          &           & {\em SWa} &           &
\\ \hline
 \SSe & {\em `Se} & {\em `Su} & {\em `Si}  & {\em `Sa}  & {\em `SE}  & {\em `S}  & {\em `So}&   &          &           &           &           &
\\ \hline
 \fe & {\em fe} & {\em fu} & {\em fi}  & {\em fa}  & {\em fE}  & {\em f}  & {\em fo} & {\em fWe}& ({\em fWu}) & {\em fWi}  & {\em fWa}  & {\em fWE}  & {\em fW}
\\ \hline
 \pe & {\em pe} & {\em pu} & {\em pi}  & {\em pa}  & {\em pE}  & {\em p}  & {\em po} & {\em pWe}& ({\em pWu}) & {\em pWi}  & {\em pWa}  & {\em pWE}  & {\em pW}
\\ \hline
\end{tabular} 
}
the names Ge'ez and Fidel; names that may be used 
through out this paper.  We start now with the 
SERA table itself and follow with a discussion of 
its derivation.  
\newpage 
%\twocolumn
 

\subsection{Considerations We Took in the Development of SERA}


We have taken the following three considerations in coming up
with our proposed standard 
\begin{enumerate}
  \item The system must be easy to type on a 101 keyboard.  This entails:
 
    \begin{itemize}
      \item finding the closest match between the Latin and Ethiopic
            phonetic systems \\ (while being as systematic as possible with
            the inevitable exceptions),
 
      \item limiting the number of keystrokes necessary for each Ethiopic
            character to a minimum, and 
    \end{itemize}
 
  \item   The system must be simple for humans to read without special
          decoders.  This requires a flexibility of the phonetic mappings
          to accommodate differing writing practices of various language 
          groups.

  \item   The system must also be easy for machine transcription.  In this
          case, the systematicity of the mapping of Ethiopic to ASCII is
          exploited to make the machine transcription between ASCII and 
          Ethiopic script (in word processors, for example) as fast as
          possible.
\end{enumerate}


\subsection{Development of the System}
 

 It may first occur to one when attempting to write Ethiopic script
 with Latin letters, to represent the first 7 forms with numbers as so: 

\begin{tabbing}
\hspace{0.2in}\=Co\=nson\=ants\=:\hspace{0.2in}\= \hspace{0.25in}\= \hspace{0.3in}\=  \hspace{0.25in}\= \\
      \>\>h1   \>h2   \>h3   \>h4   \>h5   \>h6   \>h7 \\
 
      \>Independent Vowels: \\
      \>\>a1   \>a2   \>a3   \>a4   \>a5   \>a6   \>a7 
\end{tabbing}
 
 It is soon found in practice, however, that while this is a very simple
 system for representing the Ethiopic characters, it is not so pleasant to read
 or write with (e.g., ``T5n1y6s6T6l6N6'', ``a1d5s6 a1b1b4'').  This is true
 largely because our minds are not trained to associate the Latin script with
 Arabic numbers to form words.  One will soon wonder why not use the Latin
 vowel letters to denote the 7 forms of the Ethiopic characters.  This is
 where the trouble begins:  How do you represent the standard 7 Ethiopic forms
 (plus the labiovelar ``W'' forms) with only 5 Latin vowels?    \\
 
 The first step we took was to assign a punctuation mark (the apostrophe ')
 and ``I'' for the two extra Ethiopic vowels (plus ``W'' for forms 8-12).  So, 
 following phonetic guide lines we came up with the following system:

\begin{tabbing}
\hspace{0.2in}\=Co\=nson\=ants\=:\hspace{0.2in}\= \hspace{0.25in}\= \hspace{0.3in}\=  \hspace{0.25in}\= \\
      \>\>h'   \>hu   \>hi   \>ha   \>he   \>hI   \>ho \\
 \ \\
      \>Independent Vowels: \\
      \>\>a'   \>au   \>ai   \>aa   \>ae   \>aI   \>ao 
\end{tabbing}
 
 Again, after some trial use (e.g., ``Ten'yIsITIlINI'', ``a'disI a'b'ba'') we
 found that the writing can be made more readable if we used only one
 character for the pure vowel form.  Then the system reduces to:
 
\begin{tabbing}
\hspace{0.2in}\=Co\=nson\=ants\=:\hspace{0.2in}\= \hspace{0.25in}\= \hspace{0.3in}\=  \hspace{0.25in}\= \\
      \>\>l'   \>lu   \>li   \>la   \>le   \>lI   \>lo \\
 \ \\
      \>Independent Vowels: \\
      \>\>'    \>u    \>i    \>a    \>e    \>I    \>o 
\end{tabbing}
 and our sample text would look like:  ``TenayIsITIlINI'', ``'disI 'b'ba''
 which becomes a little easier to read as well as type. \\
 
 After a short time a reader is likely to find that trying to ``read a sound''
 from punctuation proves too difficult.  Our minds have been conditioned for
 too long already to skip over apostrophes when reading possessive and
 contracted words.  We introduce the principle now that whenever possible
 punctuation be avoided to represent spoken sounds and seek another alphabetic
 character to replace the apostrophe.   \\

 We find a suitable substitute in ``E'' but recognize right away the draw back
 of the extra ``shift'' required to type it.  With only a small intuitive feeling
 one will come to realize that the 5$^{\textrm{th}}$ form letters are used less often in
 writing than are 1st form.  Hence a swap between the two forms makes the use
 of ``E'' a little easier and gives us the new table :
\newpage
\begin{tabbing}
\hspace{0.2in}\=Co\=nson\=ants\=:\hspace{0.2in}\= \hspace{0.25in}\= \hspace{0.3in}\=  \hspace{0.25in}\= \\
     \>\>le   \>lu   \>li   \>la   \>lE   \>lI   \>lo               \\
 \ \\
      \>Independent Vowels: \\
     \>\>\ e    \>\ u    \>\ i    \>\ a    \>\ E    \>\ I    \>\ o  
\end{tabbing}
\noi
 and our sample text appears a little more naturally as:  
     ``TEnayIsITIlINI'', ``edisI ebeba''

\noi
It is at this point that we began to notice two problems:  
\begin{enumerate} 
      \item the 6$^{\textrm{th}}$ (or ``sadis'') form of the Ethiopic characters occurs more
            often than any other form (about a third more often), and 
 
      \item the use of ``e'' for the first vowel makes the ``look'' of some familiar
            Amharic words peculiar, and the sound association is poor.
\end{enumerate} 
\noi
 The quick solution:  
\begin{enumerate} 
      \item stop using ``I'' for the sadis (sixth form) consonants, letting the
            consonants stand by themselves, and 
 
      \item allow the use of ``a'' for the first form independent vowel with ``e'', 
            and introduce ``A'' for the 4$^{\textrm{th}}$ form independent vowel.
\end{enumerate} 
 
 
\begin{tabbing}
\hspace{0.2in}\=Co\=nson\=ants\=:\hspace{0.2in}\= \hspace{0.25in}\= \hspace{0.3in}\=  \hspace{0.25in}\= \\
      \>\>le   \>lu   \>li   \>la   \>lE   \>l    \>lo \\
 
      \>Independent Vowels: \\
      \>\>e/a  \>\ u    \>\ i    \>\ A    \>\ E    \>\ I    \>\ o \\
 
 \                                       \\
      \>Examples:                        \\
 \                                       \\
      \>  \>TEna ysTlN                   \\
      \>  \>adis abeba                   \\
      \>  \>Indemn kermachWal            \\
      \>  \>zarE Tewat suq hEjE neber    \\
      \>  \>manew smh? manew smx?          
\end{tabbing} 
 
\subsection[nothing]{Ambiguity Problem With The \\ Independent Vowel} \label{sec:amb}
 
 This system is easier to read and type, but there is still a problem.  If
 you have never before seen the word ``TEna'' how will you know if you
 are reading 2 Ethiopic characters or 4?  I.E. ``TE-na'' or ``T-E-n-a''?  This problem
 of ambiguity usually occurs because it is not clear whether a consonant
 letter is a sadis (6$^{\textrm{th}}$) form followed by an independent vowel form, or a
 syllable made up of the consonant and following vowel form.  Of course, this
 is a problem only if the reader does not know the language.  An Amharic
 speaker would not make such a mistake.   \\
 
 In another scenario, the name ``Gabriel'' can be read ``ge-b-r-E-l'' (correctly),
 or ``ge-b-rE-l'' (not quite correct, but okay when speaking fast).  Though the
 ambiguity is there, whether you interpret the Latin as showing 5 (ge-b-r-E-l)
 characters or 4 (ge-b-rE-l) makes almost no difference.   \\
 
 These conditions may not always be true, however, and the difference does
 become a big problem for word processors and computer software for
 transcription.  It is better then to insure that the characters are
 unmistakably represented; that we \emph{not} delivered \rE \ (rE) when \r\E \ 
 (r-E) is what we wanted.  To accomplish this, our decision was to recycle
 the apostrophe ' as a separator for independent vowels that appear after a 
 sadis (6$^{\textrm{th}}$ form) consonant.  Thus, we can rewrite Gabriel as ``gabr'El'' and
 modify our system, which now includes a third category, accordingly:
 
\begin{tabbing}
\hspace{0.2in}\=Co\=nson\=ants\=:\hspace{0.2in}\= \hspace{0.25in}\= \hspace{0.3in}\=  \hspace{0.25in}\= \\
    \>\>le   \>lu   \>li   \>la   \>lE   \>l    \>lo \\
 \ \\ 
    \>Independent Vowels: \\
    \>\>e/a  \>\ u    \>\ i    \>\ A    \>\ E    \>\ I    \>\ o \\
 \ \\ 
    \>Independent Vowels Following a 6$^{\textrm{th}}$ Form Consonant: \\
    \>\>l'a \>l'u \>l'i \>l'A \>l'E \>l'I \>l'o \\
    \>\>l'e \>lU  \>    \>    \>    \>    \>lO \ \ \ \ $\leftarrow$also \\
\end{tabbing}
\noi 
 If we consider now an application for the remaining uppercase vowels; ``U''
 and ``O'', we find that in some instances, as shown in the 2nd row of the 
 third category, the use of the apostrophe may be omitted without confusion. \\
 
 Likewise, the leftover Latin consonants such as B, F, J, L, M, R, V, Y, are 
 used as alternatives for their lowercase counterparts \ |this is often nice for 
 personal names. 

\subsection{Lower Frequency Letters}
 Labiovelar series 8-12 may be mapped onto ASCII with a like logic using a
 two character syllable representation beginning with the upper case ``W''.
 A complication arises in representing 12$^{\textrm{th}}$ form syllables such 
 as \kWu \ and \gWu \ which may be known to have both the 
 sounds of ``{\bf k$^w${\"\i}}\,'' or ``{\bf k$^w$u}'' and
           ``{\bf g$^w${\"\i}}\,'' or ``{\bf g$^w$u}''.
 So in SERA both will be acceptable to indicate the same letter.  \\

 The extremely low frequency characters \mYa, \rYa, and \fYa \ may be given
 by ``mYa'', ``rYa'', and ``fYa'' respectively.  Given the rarity of these
 letters ``Y'' is not deallocated as a secondary ``y''.  It is
 acceptable to require the apostrophe to separate, for example \m \ and \ya \ 
 in Roman script (as in m'Ya, or M'Ya) should the user find the uppercase
 ``Y'' construct necessary.

