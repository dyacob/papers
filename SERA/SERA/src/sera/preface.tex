
%\centerline{{\large\bf Preface}}
%\vspace{0.25in}
%
% This is utterly raw, it will be reworked appropriately
%
\section*{Preface}
In the time since the original publication of our paper
in \emph{The Journal of EthioSciences} ~\cite{EthioSci} on the
topic of representation of Fidel in 7-bit ASCII, the need
became apparent to extend the system to encompass
representation for Ethiopic numerals, punctuation, and mixed
script notations.  In the same period more was learned about
the treatment of certain characters outside of Amharic that
allowed for simplification of the ASCII representation.  The
following is a recapitulation of the original publication and
an assessment of some of the more recent developments.  \\

As we have indicated before, this system, though well
developed, is still not in its final form.  Further refinements
will only come after many have had the chance to use it and
test its strengths and weaknesses on their own.  As the SERA
system, joined now by a long awaited Unicode~\cite{Uni2} domain assignement, 
work together to advance Ethiopic information processing
we keep in mind the words of Abraham Demoz (1925 - 1994), to whom 
we have dedicated this work ~\cite{Demoz}:
\begin{quote}
	``\ldots script reform calls not only for a competent
	professional assessment of the technical aspects of
	the script but also for a careful weighing of these
	against the psychological and socio-political factors
	that have a bearing on the written word and all that
	it stands for''
\end{quote}

% This will be added to the bibliography later.
%
% (Demoz, "Amharic Script Reform Efforts".  ETHIOPIAN
% STUDIES. S. Segert and A.J.E. Bodrogligeti, Eds. 1983).

% Any and all feed back will be greatly appreciated.


% \begin{tabbing}
% \ \ \ \ \ \ \ \ \ \ \ \ \ \ \ \ \ \ \ \= \kill
% \>dan'El yaqob (Daniel Yacob) \\
% \>yTna frdyweq (Yitna Firdyiwek)
% \end{tabbing}



%
%  Represent the 1994 SERA document with thought from the FAQ added.
%  Break paper into sections of letters, numbers, and punctuation.
%  This presents SERA completely and sufficiently.
%
%  Then add in appendix form extra info for keyboard considerations and
%  technical aspects.
%
