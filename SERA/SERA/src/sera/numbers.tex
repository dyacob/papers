
\section*{Ethiopic and Arabic Numbers in SERA}
   
The Arabic and Ethiopic numerals will both be given with the Arabic 
numbers found on Latin keyboards.  The Arabic numbers, the more common,
may be used in the usual way; Ethiopic numbers 
require the SERA alternate specifier, ` , before the number.  An 
understanding of the Ethiopic number system will benefit the composer. \\

\noi
SERA uses a ``Fidel For Numbers'' of sorts where zeros play the role of vowels.
This system allows for a simplification in the writing of Ethiopic numbers. \\

\hspace{-0.20in} `10`9`100`80`7 = `109100807 = `10900807 = \asr\zeteN\meto\semanya\sabat \\

\noi
%\hspace{-0.5in}\parbox{3.5in}{
\begin{tabular}{|c|c|c|c|c|c|} \hline
        & ones   & tens   &  hundreds & thousands & ten-thousands \\ \hline
        & \and   & \asr   &  \meto    & \asr\meto & \asrxi        \\ \hline 
        &        &   0    &  00       &  000      &  0000         \\ \hline
 \and   & `1     & `10    & `100      &`1000      &`10000         \\ \hline
 \hulet & `2     & `20    & `200      &`2000      &`20000         \\ \hline
 \sost  & `3     & `30    & `200      &`3000      &`30000         \\ \hline
 \arat  & `4     & `40    & `400      &`4000      &`40000         \\ \hline
 \vdots & \vdots & \vdots & \vdots    &\vdots     &\vdots         \\ \hline
 \zeteN & `9     & `90    & `900      &`9000      &`90000         \\ \hline
\end{tabular}%} 

