

\section*{Punctuation and SERA Special Characters} 

SERA uses three special characters for writing Fidel in ASCII documents.
Backslash, $\backslash$ , is used to begin and end blocks of text written in Fidel or other
writing systems such as Roman, Arabic, Hebrew, etc.  Backslash can also separate
blocks of text written in the same writing system but in different languages
-this is useful to insure satisfactory transcription under different rules that
languages can follow.  Script and language separation is backslashes primary
purpose.  We will see shortly that it can also be used for additional special
purposes called ``escapes''. \\

Apostrophe, ' , also called single quote, serves the purpose of separation in SERA.
Apostrophe appearing in an Ethiopic block of SERA text will not appear in the
transcription into real Fidel.  Apostrophe is usually used after a 6th form consonant
when a vowel follows.  This avoids confusion for computers so that r'E can only turn out
in Fidel as \r\E \ and never \rE \ (which ommiting the apostrophe produces).  Another use
of apostrophe is between two vowels to make the ASCII reading a little clearer such as
ke'ityoPya instead of keityoPya.  This use is left as a typist preference and makes no
difference on the final outcome. \\

Backquote, ` , also known as ``spacing grave'', usually has the function of providing
an ``alternate'' of the token that follows.  Examples are the letters series `s, `h,
`e, and `S for \sse, \hhe, \ee, and \SSe \  respectively.  Also punctuations and numbers in
Ethiopic writing that have alternatives in purpose such as `?, `: and numbers `1\dots
`10000. \\

\noi
{\large\bf Ethiopic Punctuation} \\
\begin{tabular}{|c|c|} \hline
   ,             & \Gcomma                       \\ \hline
   ;             & \Gsemicolon                   \\ \hline
   :-            & \Gprecolon                    \\ \hline
  -:             & \Gcolon                       \\ \hline
  `:             & \Gspace                       \\ \hline
   :             & {\begin{tabular}{c|c} \Gcolon \   $\,$ & \ \ \Gspace \end{tabular}} \\ \hline
   ?             & {\begin{tabular}{c|c} \QMARK  \ \ $\,$\hspace{0.01in} & \ \  \Sost  \end{tabular}} \\ \hline
  `?             & {\begin{tabular}{c|c} \Sost   \ \ $\,$ & \ \  \QMARK \end{tabular}} \\ \hline
  ::             & \Gperiod                      \\ \hline
  :$|$:          & \Sabat                        \\ \hline
  $<<$           & \frlquote                     \\ \hline
  $>>$           & \hspace{-0.08in}\frrquote     \\ \hline
  ''             & {\bf\"{\ }}                   \\ \hline
  `'             & \parbox{0.75in}{\tiny{``Vocalized Sadis'' control character \\ \hspace*{0.12in}for linguists}} \\ \hline
  `\ '           & {\tiny Ignored If Alone}      \\ \hline
  `1\dots`10000  & \and\dots\asrxi               \\ \hline
\end{tabular} 
\hspace{0.2in}\parbox{2.50in}{The defaults for : and `? / ? use are presumed set by a user in the software
using SERA.  The defaults may be toggled to the alternative setting at any point
with $\backslash\tilde{}\,$: and $\backslash\tilde{}\,$? .  \Sost \ is the suggested default
for `?  use while no default is suggested for colon.   It is useful to have fixed ASCII definitions for
\Gspace \ and \Gcolon \ (namely `: and -:) while : is available for one or the other.  It is not 
expected that all  three defininitions would be found simultaneously in a given block of text.  To ensure the 
correct use of : and ? when exporting a document, the settings may be recorded at the start of the document with
$\backslash\tilde{}$\{set : `:\} \\ (or $\backslash\tilde{}$\{set : -:\}) and $\backslash\tilde{}$\{set ? `?\}. } 

\vspace{0.5in}
 
\noi
{\large\bf Escapes} \\

\noi
Escapes are universal in any region of text. \\
\noi
\begin{tabular}{|l|l|} \hline
  $\backslash$   &   Script Toggle ... $\leftarrow$ fidel $\rightarrow$ $\backslash$ $\leftarrow$ latin $\rightarrow$ $\backslash$ $\leftarrow$ fidel $\rightarrow$ etc. \\
                 &   One or more punctuation characters following $\backslash$ will not \\
                 &   require a closing $\backslash$  The script toggle terminates when the \\
                 &   first nonpunctuation charcter is reached. \\ \hline

  $\backslash\backslash$  &   $\backslash$ The backslash escape is also in following with normal \\
                &    rules for punctuation. \\ \hline

  $\backslash$! &   Escapes are ignored until closing $\backslash$!  \\ \hline

  $\backslash\tilde{}\,$xxx &  \\
  $\backslash\tilde{}$\{do xxx\}    &       Perform ``xxx''  \\ \hline
\end{tabular} \\

\noi
A space `` '' following the above escapes will be removed from the transcribed
output. Space at the end of a punctuation list after $\backslash$ ( as in $\backslash$$\backslash$ or $\backslash$:$|$: ) will
be preserved.

