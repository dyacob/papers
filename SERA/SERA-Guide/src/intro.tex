\centerline{\LARGE\bf The SERA User's Guide} 

\vspace{0.4in}
``SERA'' is an anacronym for ``(The) System for Ethiopic Representation in ASCII''. Most 
simply put -SERA is a way to write in Fidel script using Latin letters. More 
extendedly; SERA is a convention for transcription of Fidel script into Latin that 
insures the integrity of the format and content of the original document, and that 
it be fully transportable across all computer mediums.  \\

As important as the preservation of the original content; the transcription system 
is also designed to be as easy to read and type as possible. SERA has been under 
continued development since early 1993 with the aim to fill these roles naturally 
and intuitively. \\

SERA is a convention for all elements of Ethiopic text which includes; letters, punctuation, and numerals.
SERA also provides a mechanism for multilingual writing and the embedding of special control characters for
your favorite software. \\
