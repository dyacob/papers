

\section*{Punctuation And SERA Special Characters} 

SERA uses three special characters for writing Fidel in ASCII documents.
Backslash, $\backslash$ , is used to begin and end blocks of text written in 
Fidel or other
writing systems such as Roman, Arabic, Hebrew, etc.  Backslash can also separate
blocks of text written in the same writing system but in different languages
-this is useful to insure satisfactory transcription under different rules that
languages can follow.  Script and language separation is the primary purpose of
backslash.  We will see shortly that it can also be used for additional special
purposes called ``escapes''. \\

Apostrophe, ' , also called single quote, serves the purpose of separation in SERA.
Apostrophe appearing in an Ethiopic block of SERA text will not appear in the
Ethiopic transcription.  Apostrophe is usually used 
after a 6$^{\textrm{th}}$ form consonant when a vowel follows. 
Another use
of apostrophe is between two vowels to make the ASCII reading a little clearer such as
ke'ityoPya instead of keityoPya.  This use is left as a typist preference and makes no
difference on the final outcome. \\

Backquote, ` , also known as ``spacing grave'', usually has the function of providing
an ``alternate'' of the token that follows.  Examples are the letters series `s, `h,
`e, and `S for \sse, \hhe, \ee, and \SSe \  respectively.  Also punctuations and numbers in
Ethiopic writing that have alternatives in purpose such as `?, `: and numbers `1\dots
`10000. \\

\noi
{\large\bf Ethiopic Punctuation} \\
\begin{tabular}{|c|c|} \hline
   ,             & \Gcomma                       \\ \hline
   ;             & \Gsemicolon                   \\ \hline
   :-            & \Gprecolon                    \\ \hline
  -:             & \Gcolon                       \\ \hline
  `:             & \Gspace                       \\ \hline
   :             & {\begin{tabular}{c|c} \Gcolon \   $\,$ & \ \ \Gspace \end{tabular}} \\ \hline
   ?             & {\begin{tabular}{c|c} \Gqmark \ \ $\,$\hspace{0.01in} & \ \  \Sost  \end{tabular}} \\ \hline
  `?             & {\begin{tabular}{c|c} \Sost   \ \ $\,$ & \ \  \Gqmark\end{tabular}} \\ \hline
  `!             & \slaq                         \\ \hline
  ::             & \Gperiod                      \\ \hline
  :$|$:          & \Sabat                        \\ \hline
  $<<$           & \raisebox{-0.2ex}{\Lquote}    \\ \hline
  $>>$           & \raisebox{-0.6ex}{\Rquote}    \\ \hline
  ''             & \Ggeminate                    \\ \hline
  `'             & \parbox{0.75in}{\tiny{``Vocalized Sadis'' control character \\ \hspace*{0.12in}for linguists}} \\ \hline
  `\ '           & {\tiny Ignored If Alone}      \\ \hline
  `1\dots`10000  & \and\dots\asrxi               \\ \hline
\end{tabular} 
\hspace{0.20in}\parbox{2.50in} 
  {\vspace*{-0.7in}The defaults for : and `? / ? use are presumed set by a user in the software
    using SERA.  \Sost \ is the suggested default for `?  use while no default is suggested for colon.  
 Defaults may be reset to the alternatives at any point; $\backslash\tilde{}\,`|$ would
 set ? for \Sost \  which could be reset by $\backslash\tilde{}\,$? \.
 It is useful to have fixed ASCII definitions for
 \Gspace \ and \Gcolon \ (namely `: and -:) while : is available for one or the other. Colon
 usage may be set and reset with $\backslash\tilde{}\,$-: $\backslash\tilde{}\,$`: \ .
 To ensure the correct use of : and ? when exporting a document, the settings may be recorded at 
 the start of the document.}


%\vspace{0.5in}
\newpage
 
\noi
{\large\bf Escapes} \\
%\subsection*{Escapes}

\noi
%
%  Not in SERA-97
%
% Escapes are universal in any region of text. \\
%


The core of SERA will always be its transliteration definition for the Fidel
syllabary. SERA provides ``escapes'' or ``switches'' so that changes of
language and scripts can be signaled to a reader without requiring special
software to read the document. Special purpose escapes are also provided so
that applications may communicate graphic elements and processing directives in
an ASCII document. \\

The backslash character then is chosen for escapes in SERA as it is in
agreement with the existing conventions of Unix, \LaTeX, C, and other
programming languages. Basic escapes are:\\


\noi
\begin{tabular}{|l|l|} \hline
  $\backslash$   &   Script Toggle ... $\leftarrow$ fidel $\rightarrow$ $\backslash$ $\leftarrow$ latin $\rightarrow$ $\backslash$ $\leftarrow$ fidel $\rightarrow$ etc. \\
                 &   One or more punctuation characters following $\backslash$ will not \\
                 &   require a closing $\backslash$  The script toggle terminates when the \\
                 &   first nonpunctuation charcter is reached. \\ \hline

  $\backslash\backslash$    &   $\backslash$ The backslash escape is also in following with normal \\
                 &    rules for punctuation. \\ \hline

  $\backslash\tilde{}\,$!   &   Escapes are ignored until closing $\backslash\tilde{}\,$!  \\ \hline

  $\backslash\tilde{}\,$xxx &  \\
  $\backslash\tilde{}$\{do xxx\}    &       Perform ``xxx''  \\ \hline
\end{tabular} \\

\noi
Whitespace is the required terminator for all escapes. When space, `` '',
is the following whitespace terminator it will be removed from the transcribed
output. 



%\noi
%A space `` '' following the above escapes will be removed from the transcribed
%output. 
%
%  Not in SERA-97
%
% Space at the end of a punctuation list after $\backslash$ ( as in $\backslash$$\backslash$ or $\backslash$:$|$: ) will
% be preserved.

