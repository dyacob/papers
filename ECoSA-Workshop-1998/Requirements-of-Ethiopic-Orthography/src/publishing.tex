The earliest example of what today would be recognized as Ethiopic writing is found chiseled in stones that have managed to survive through the ages.  While this technology has so far proven more durable than anything that has followed since, it was laborious to prepare the tablettes for writing and correcting mistakes might often have meant nothing short of starting over. \\

Brana as a competing technology required a more detailed and lengthier process to prepare the writing surface but offered improvements in weight, portability, rendering and corrective labor.  The ease of use of the new medium offered other advances to publishing such as the proliferation of typefaces, greater use of images and the embedding of images with text, also the use of colored text for emphasis.  Writing could become available to greater numbers of people and so the written language and orthography was able to mature.  The increased use and number of punctuation marks for instance is an obvious advancement.  The first postal system was able to emerge in the age of Brana.\\

Paper would later begin to displace brana when the production costs became economically favorable.  The Ethiopic printing press of Abebe Giorgis and Ludolfi in the 16$^{th}$ century marked the start of a new age of publishing in Ethiopic.  With the mass reproduction capabilities that the printing press offered to paper media the age of brana ebbed and new forms of publishing gave rise.  Periodical literature flourished, the scribner would be displaced, publishing house rose and typesetting became a competitive art.\\

The telegraph and fax machine were invented in the age of the typewriter.  The mimeograph and photocopy machines would also bring small scale publishing into the office place as document reproduction no longer required the hardware and labor investment of a printing press.\\

The Internet was invented in the age of the computer, as one made possible the other.  For the distribution possibilities the Internet offers it has been called the most significant new invention in publishing since the printing press.  
% Greater incorporation of visual media was possible as well as incorporation 
New media such as video and audio could become a part of document publishing for the first time though requiring publishing to be strictly electronic.  When computer composed text was published in print, likely still at a printing press, the corrective features a computer could offer for spelling and grammar would raise the level of publishing professionalism as a whole.\\

Interestingly enough all of these technologies remain in use today.  Stone engravement and brana do not compete with the paper and electronic mediums -they are accepted as their own form of art and of special purpose use.  As computer cost have come down they replace typewriters more and more each passing year as composition and rendering tools in the printed mediums. Typewriters do not compete with computers in electronic mediums.  The printing press as a mass production tool of printed media has to vie now with the computer as a tool that renders text on a writing surface that is itself the medium of the masses.  In our age these tools and medias have come to compliment one another.  It is much too soon to assume paper will go the way of brana and stone, precedence at least indicates that this would not happen in the time scale of a single human life span.
\newpage
The trend that determines the success of each new advancement is the increased capability to replicate and distribute documents.  What we have lost along the way, starting from brana, is some of our ability to compose documents as freely as we did with the previous medium or tool.  This is in large part an issue of the economy of the technology and a smaller part of the technologies actual limitations.\\

When the hand is our tool for rendering the strokes of letters the calligrapher has a great liberty to do as he or she wishes within his or her own's limitations as an artist.  Not only could art or text be oriented at any particular location but there would be no limitation to the number of scripts that could the composer could use.  Authors could even create new phonetically meaningful characters or symbols as needed.  With the printing press these flourishes of the calligrapher were traded for the speed of text reproduction. Publishing would be limited to the set of tiles that had been made for the particular printing press.  Same too for the typewriter, though it offered some capability to compose new letters.  With the computer the capability to incorporate illustrations returned, but we continue to struggle to use computers in a natural way with Ethiopic script.\\


Suppose a new publishing device comes along that offers us an improvement in document reproduction, distribution, or print quality or all of theses.  What do we need to know about Ethiopic to use it with the tools of any medium?  Firstly, we would want to use the new apparatus with all of the typesetting conventions that had developed over the years.  Secondly, we want available to us all of the written elements ever in use by Ethiopic publishing since its inception and we will need them in every size imaginable.  Thirdly, we would probably like to work with other scripts as well as art work without incurring additional complications.  \\

In the ideal use of a new publishing invention we would want to sacrifice none of what we were capable of producing previously before reaping the benefits of the new technology.  The computer should be an ideal device preserving and even restoring past practices while opening the doors to newer possibilities that we are just beginning to be able to dream of now.  Perhaps it is unexpected then that those who would begin to colonize these brave new electronic frontiers for Ethiopic were not experts in typography and the publishing arts.  Indeed, perhaps in part because the rise of computer use came when Ethiopia was in a time of social and political turmoil that Ethiopic electronic publishing would develop without the benefit of a detailed exploration of Ethiopic publishing practices through time and space.\\

In the remainder of this paper we will attempt such an exploration to uncover contemporary and historical writing practices in Ethiopia.  We turn first to examine those languages that have at some point in their literal tradition used the Ethiopian writing system.\\

