
\subsection*{Languages and Orthgoraphy}

Languages of Ethiopia which are numbered just more than one hundred are
classified into four language families.  Namely, Ethio-Semitic, Cushitic,
Omotic, and Nilo-Saharan.   From these hundred and more languages only 
seventeen of them are used as school subjects and media of instruction in 
primary schools of Ethiopia.  Amharic, Ormomo and Tigrigna have been used
since the 1950 and 1970s in schools.  The others were introduced as languages
of literacy as of the mid 1970s.\\

These seventeen and other languages at different times have used three
types of scripts; Ethiopic, Latin and Arabic.  Presently Ethiopic script,
which is our concern, is used by Amharic, Tigrigna, Argobba, Siltie, Soddo, 
Agew, Ari, Harari, and K\"{a}bena languages.\\

Arabic script has been used in Gambella and Beri-Shengal regional states in 
the past.  Muslim scholars, who are composers of poetry, specially in Wello
use Arabic script to write their poems in Amharic.  It is called Ajam.\\

Latin script in the majority of cases have been used by cushitic languages
like Oromo, Sidama and Omotic languages Wolayyta and others.\\

The choice of the script for languages prior to 1991 have been made by the
successive governments.  After 1991 it is usually by the people or the
politicians who represent these people.\\

As the concern of this workshop is to enable the users of Ethiopic script to 
benefit from the technologies by incorporating it in the computer.  It
focuses on 1) languages that have used this script in the past now shifting
to Latin script.  2) Languages that are using this script in the past and
present, and 3) some ligatures and Ethiopic punctuation marks are also 
included.\\

For the time being based on the available documents, the ones described
here 1) Those who used Ethiopic in their literacy campaign Sidama, 
K\"{a}mbatta, Wolayyta, Ari, Kefa, Afar, and Saho.  And 2) from those who use
Ethiopic script at present; Amharic, Agew, Siltie, Tigrigna, K\"{a}bena and
Ari. 3) the other languages like Konso, Gumuz, Dizi, Bench and Yem.\\

Most of the orthographics have been found from the former Ethiopian
Languages Academy.  The others are from student theses, missionaries and
from the people who worked on them.\\

% The presentation of the orthographics is done in two ways.  One way is putting
% the alphabet (or the fidel) under language names and two using a table putting
% the alphabet in the chart with their corresponding phonetic forms.

The presentation of the orthographics is done by putting the alphabet in the
chart with their corresponding phonetic forms.\\
