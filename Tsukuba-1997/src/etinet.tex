


%            Transliteration on the Internet: The Case of Ethiopic
%                                    by
%                           Daniel Yacob Mekonnen


\section*{Introduction}
In the later half of this century we are experiencing what has already for
some time been dubbed the ``Information Age''.  For many of us in this era
the ability to exchange electronic text shapes the way we live, work, and
even recreate.  
Unlike the printed mediums the information we receive electronically  we can
easily manipulate and repropogate.  The variable of physical distance is all
but eliminated to users in the electronic communication formula.\\


%This ability eliminates the variable of physical distance
%between points from the act of communication.  Unlike the printed medium the
%information we receive in the electronic mediums we can easily 
%manipulate and repropogate.  The communication grows.\\


% One aspect of The Information
% Age is emergent from society's incorporation of computers and other electronic
% devices (e.g. faxes, telex, personal assistants) as the norm for text based
% communications.  Unlike the printed medium the electronic form of text is more
% readily transmitted over long distances, manipulatable by the reader and as
% easily retransmittable as it was received.\\

But this is only readily done for scripts of the developed world.  In the
developing world where the indigenous writing systems did not play a part
in the computer revolution, societies and developers are now scrambling to cope
with script interchange issues as the information age advances upon them.  As
if it were an elemental force in nature, two people at any distance apart in
a common medium will be drawn towards each other to communicate with whatever
means the medium allows.  The ``communicational potential'' between them will be
too strong for exchange not to happen; they will communicate, information will
propagate.\\

In this article we will look at what happened when the information revolution
began to subsume Ethiopic\footnote{``Ethiopic'' is an ISO term to be certain,
native users prefer the terms ``Ge'ez'' ({\smallet \char174{\smalletx\char138}\char132}) or
``Fidel'' ({\smallet \char165\char100\char204}).}
script -the writing system used in present day Ethiopia and Eritrea.  We will
look briefly at the issues for why Ethiopic was caught unprepared as the
Information Age dawned and what happened when the forces of communication were
greater than the resistance of the medium to Ethiopic information interchange.


\section*{Background}
The modern Ethiopic writing system is now roughly 500 years old.  The origins
of the script trace back through the Mino-Sab{\ae}an and ultimately lead to the Phoenician.
The precursors to the modern script are found in Eritrea and Ethiopia as early
as 1,000 BC.  The script did not begin to resemble its modern form until the mid
4th century AD following the arrival of Syrian missionaries who would translate
The Bible into the then dominant language Ge'ez.  The writing direction would be
reversed from right-to-left plow style to left-to-right regular style.  The script would
also become a syllabary.\\

The modern day script contains writing elements for twenty numerals, ten unique
punctuation marks, and a syllabic domain of, to this day, uncertain size.
Also noteworthy is that the script has only a single case and a cursive form
is not used. \\

Syllabaries are generally of a known and manageable size, so it may come as some
surprise the discovery that Ethiopic's full extent is still veiled in mystery.  To
be sure, few would argue over the elements of the stable syllabary of the last 5
centuries.  This is often referred to as the ``Amharic Syllabary'', for the
extensions upon the Ge'ez syllabary made for the literal practices in the language
Amharic.  The overwhelming mass of Ethiopic literature is in the languages Amharic,
Tigrigna, Ge'ez, and lesser so in Oromigna, whose speakers have long literal
traditions.  The combined syllabary for these languages holds 36 consonant series
in 7 syllables each.  Most consonant series also have an 8th form and 6 series
have a complete set of 12 forms for a total of 307 syllabic letters.\\
%
% Describe "potential" state of 12?
%

Of the no fewer than 82 languages spoken in the combined Ethiopia and Eritrea,
many societies have only in this century entered into a literal tradition from
an oral one.  In so doing they would encounter that the Amharic syllabary would
not be sufficient for representing all of their spoken sounds.  New letters
would then be introduced following the syllabary's intrinsic convention for the
introduction of new syllables.\\

Writing extensions however were usually devised and offered to these peoples.
Establishing that they have accepted the characters and made them a part of
their present day writing practices is a major barrier to current 
standardization efforts (i.e. ISO-10646).



\section*{Foreground}
When Ethiopic script first ventured onto the personal computer is uncertain but
is likely to have occurred in late 1985.  The Ethiopian Science and Technology
Commission was behind early breakthroughs and soon formed the National Computer
Center to
%
%  Ask Daniel if "National Computer Center" is correct or was it NCIC?
%
lead computerization efforts.  At the center's peak DOS 3.2 was disassembled and made
completely Ethiopic.  The same approach was repeated for at least ten popular
softwares as work on new offerings developed by the center began to move forward.\\

Tragically, given the volatile political climate of the day, the Ethiopian
Science and Technology Commission would not be in a position to standardize and
direct the migration of Ethiopic script into the computer world.  Most of the
notable work that would follow the initial triumphs would happen outside of
Ethiopia.  Ethiopians in dispora, driven my hobbyist or commercial interest
would repeat the first accomplishments and move forward with each new wave of
progress by the computer industry.  They would offer new software and adapted
existing software to support Ethiopic.  The ESTC would not keep pace with the
industry and the pioneering  work was soon antiquated\footnote{
\emph{The truly interested can still purchase from the ESTC ``Agafari'' the
Ethiopic DOS operating system}}.\\

Without a central body at the core of this new industry progress made by
individuals went on in isolation and in varied directions.  There would be
little, if any, communication between these pioneers who were now becoming
market competitors.  The field would grow and each new vendor would use his
own character coding systems for Ethiopic.  Different solutions would be
applied to manage the more than 320 Ethiopic elements into the confines of the
less generous ANSI.  Some would apply multiple font systems, others would break
Ethiopic into base forms and a series of diacritic marks in order to use
a single font.  The lesser used characters might be dropped from the syllabary,
the lesser responsible developers would try to introduce completely new
writing elements.  In the more than 25 coding systems in use today; the range
of characters available goes from a paltry 200 to an unlikely and bloated 480.
None of the solutions would be satisfactory, and none of the solutions would be
compatible.\\

This chaos of ever increasing character coding systems, exploding computer
proliferation and numbers of computer users, would be the setting for the
head-on collision that would happen next:  The Internet.\\


\section*{The Need To Communicate}
In the early 1990's Ethiopian (and now Eritrean) users of popular Internet
communication services, such as email and network news, grew in number until
in December of 1991 the first email list server was established for group
communication within this community.\\

The problem of communication between varied computer systems, softwares,
file and text formats was never more apparent.  Users would go to the extremes
of mailing uuencoded images of documents in desperate attempts to communicate
in Ethiopic.  Ultimately the community would have to work with what was 
available to everyone:  Qwerty array keyboards with only Roman script to read
and type.\\

The complications of communication in Ethiopian languages with a foreign script
would become apparent, and dynamically so.  With each consecutive email
posting, and even within the body of the same message, new renderings of
Amharic words could be found.  Without reservations authors would mix different
rules from English's confusing orthography and sometimes with the Ethiopic
keyboard mappings that their favorite software applied.\\

Subscribers began to copy one another's conventions and before long in this
interface between a natural language and a foreign script there was a gradual
convergence upon a transliteration creole of sorts.  There would be less
variance in word renderings but absolute consistency escaped, the arrival of
new subscribers would also perturb the creole's convergence.\\

There was an underlying regularity that was invariant, particularly
concerning the use of consonants.  Truly, there was not an overwhelming concern
for the issue among the regular subscribers, members were more or less
content to make cognitive inferences to reconcile the lax writing practices.\\

A few developers however saw the potential for what a regular and consistent
system would offer.  With a regular system these ASCII email files could easily
be translated by computer software.  The possibility was at hand for users to
use their Ethiopic software to import and export email messages.  While it
would be inconvenient to have to use two software package to read Ethiopic
email, it would be a first step.  More importantly a common ground would be
introduced for Ethiopic text interchange, a system that was not tied to any
one vendor, a system that was human readable and did not require special
software to compose, an intuitive system that people were more or less using
already.\\


\section*{Transliteration}
Work began on a formal and rigorous transliteration system that would later
become ``The System for Ethiopic Representation in ASCII'', or SERA.  The
issue of representing Ethiopic text in a conversion alphabet was nothing new.
The application and constraints of the problem in electronic media however,
were probably being faced for the first time.\\

Before going further it is useful now to address exactly what transliteration
is and what it is not. It has been the author's experience that some amount of
confusion surrounds the use of the two terms ``transliteration'' and
``transcription'' -related but critically different practices in the conversion
of a system of writing.\\

The best way to clarify the difference is with an example;  lets illustrate
the two with the salutation ``{\seG}{\laG}{\mG} {\gWaG}{\deG}{\NoG}{\cEG}'' for ``Greetings my friends''.
In a transliteration system our goal is simply to map character for character
the Ethiopic script to a target syllabary or alphabet such as Latin.  The
result in SERA would surface as ``selam gWadeNocE'', or in the working
ISO/TC46/SC2 standard which applies only the single case 
``selam gwade`noche{'}\,''.  If we are careful to restrain our mappings to being
1-to-1 we also create a reversible system and can retransliterate the Latin back
into Ethiopic without loss of content.  Both SERA and the ISO standard are reversible
transliteration systems.\\

It is not of necessity that a transliteration system attempt to preserve the
spoken level of the text elements but it is usually practical to do so.  For
instance, were we to consider only the transfer of text elements we could
apply a transliteration system that marked syllables by their numeric order and
render ``s1l2m6 g8d1N8c5''.
The numbers however are considerably less natural to read than a system
applying vowels.  At the other extreme is ``s\"{a}lam g$^w$ad\"{a}\~{n}o\v{c}\={e}''
under a phonetic system.  The phonetic alphabet goes further to preserve the
spoken level (minus gemination) but would also be unnatural to the average user.\\

Transcription goes a step further and considers not just the target script but
the writing practices of a target language that uses the target script.  A
transcription system will apply capitalization to proper nouns, doubling of
geminated words, and inflict the ``norms of irregularity'' upon the transcribed
words in the orthographic conventions of the target writing system.  For 
instance our example phrase in American or British English might render as
``selam gwaddegnochay'' there would be little debate concerning the transcription
of ``selam'' which happens to also match its transliteration.  ``\,{\NoG}\,'' however may
take the forms ``ng'', ``gn'', and ``ny'' probably without doubling of one of the two
radicals -though transliterated into Spanish as ``\,\~{n}\,'' would certainly be doubled.
``\,{\cEG}\,'' could transfer as ``chay'', ``chei'', ``chie'', ``chae'', ``che'', ``chai'', and
perhaps even ``cheigh'' with more or less mutual acceptance.\\

It is important to point now out that a transcription system is also \emph{not} necessarily
a 1-to-1 mapping as the above example helps illustrate.  Rather, such a transcription
system would be an exceptional (or trivial)  case.  Other complications have crept
into Ethiopic.  In Amharic, the largest spoken language using Ethiopic script, there
are 4 syllabic series in the phoneme base of ``he'' ({\heG}, {\HeG}, {\hheG}, and {\KeG}) that would all map
onto the English ``h'' plus a vowel.  In Tigrigna, another language using Ethiopic, there
are only two series in ``he'' ({\heG}, {\hheG}) that map onto ``h''.  In Ge'ez, the
language for which the script has its namesake, only {\heG} would map onto ``h''.\\

Again we see transcription will fail to be reversible but it is not applied
for that purpose.  At the time the need was found to devise SERA a nearly 30
year old system for the transcription of Ethiopic text was in use by the
Institute of Ethiopian Studies in Addis Abeba (though applying doubling the
system is erroneously referred to a ``transliteration'').  The IES system is used
largely for bibliographic purposes and cataloging of the library's collection
and less frequently for the transcription of a complete manuscript.\\

The IES transcription system would not be able to solve the problem of
electronic text interchange for Ethiopic.  In addition to the reversibility
problems discussed; the IES system is not typographically possible with a
standard PC keyboard.  Further, many elements would require
8 bits for interchange or are not found in the IS0-8859-1 or ISO-8859-2 sets.
The last failing of a transcription system for the average Ethiopian user was
that it would require a very strong command of English.  Transcription of many
words would be a challenge even to native speakers!\\
SERA would have to be simpler for the average user to learn and apply, it would 
have to be a transliteration system.  Initial considerations were also that
SERA be easy to type and read as well as implement readily in transliteration
and retransliteration algorithms with the basic ``macro'' languages used by
software of the day.  Since text had to be transferred through all existing
Email gateways it was also a requirement that the system apply only 7 bit
characters.  Finally, as it was also the prevalent practice in email exchanges
to mix English with Ethiopian languages SERA would also have to provide a
mechanism to indicate a change of scripts.\\


\section*{Of Utility and Things To Come}

In the experience of the application of SERA over the last 4 years it has been
the exception rather than the norm that it be used in the email exchanges for
which it was devised.  Except for the occasional exchange of sizeable length,
users generally do not make the effort to write in the more rigorous manner that
SERA demands.\\

The mechanism of transliteration to escape the volatile character code
realities of Ethiopic has in several instances been realized.  Users of
Multilingal Emacs' (Mule's) ``Ethiopic Mode'', which applies SERA, will not
notice that the Ethiopic font has changed to the working Unicode/ISO-10646
specification.  No documents will be effected.  Mule also applies SERA based
escape sequences for Ethiopic \TeX~support.  This is of some very high
practical value when considering the migration of the \TeX ~community to Unicode.
%Documents saved in the escape system will not suffer any damage to content when
%used in the newer systems.\\
%Demonstrating this imperviousness to character codings is 
%
%  this is a bad sentence....
%
%the Ethiopic support now available for the \TeX~\Babel~package applies a
Before then, Ethiopic \TeX ~users may wish to start using the newly released
\ethioplogo~package for Ethiopic support in \Babel.  The \ethioplogo~package
applies a secondary transliteration system as well as a new character coded 
font.  Again, the qualitative Ethiopic text for Mule users remains impervious
to these changes.\\

The need to handle Ethiopic in 7 bit environments has given SERA additional
application areas that were never expected.   As an input method
for Ethiopic text SERA (and a modified version known as SERA-IM) have been
applied in Ethiopic versions of UNIX ``talk'', IRC-Chat, a web based ``Chat
Room'' interface, text editors including Mule, and one commercial product.\\

Finally, and most unexpectedly, for Ethiopic document Optical Character
Recognition (OCR) by one researcher in Czechoslovakia.  When his Macintosh
operating system could
not provide a way to store the addresses of Ethiopic character glyphs, SERA
mapping were used instead.  This growing collection of work has been online as
the ``Library of Ethiopian Text''.  Browsing world wide web with the Mule
editor this collection is transliterated automatically for the viewer. A
Java plug-in for Netscape 4.0 will do the same for PC users and transliterate
the SERA web pages into the character codes of the  users's  favorite font
package.\\

In the reverse direction the current text only Lynx browser will convert UTF-8
encoded text into SERA for those who are not able to display UTF-8 Ethiopic
text in their terminals.  On the horizon in 1997 is a ``DLL'' for Ethiopic
information processing to assist vendors in applying SERA.  A Java ``Beans''
implementation of a SERA lexer should also be seen shortly. \\


While promising efforts seem to be underway for SERA support in more and more 
applications, new solutions for Ethiopic text transfer may be coming forth that
would eliminate the need for a transliteration system altogether.  Almost all
major operating systems will begin to support Unicode and UTF-8 encoding by the
end of 1997.  UTF-7 support should become more prevalent in software as well,
allowing for the transfer of Ethiopic text across the same mediums that SERA
was designed for.\\

The text world will not be viewed through UTF-7 and 8 lenses overnight of 
course.  But lets consider a time, however distant, when this might be true.
The cost of using Unicode in UTF-8 is an expensive one to the Ethiopic users.
It is true that the Unicode/ISO-10646 specification remains tentative, the
basic domain however is unlikely to change when it has passed its last stage of
officiation.  The working specification for Ethiopic puts it in the range where
it requires 3 bytes per character under UTF-8 encoding.  File sizes are then
150\% larger than that of most European scripts and 300\% more than that of
the 7 bit range.\\

In an analysis of a document containing more than 10,000 Ethiopic
characters we find that to store the file with SERA text requires roughly 79\%
of the bytes of a 2 byte per character system, and 53\% of the bytes needed
by UTF-8 encoding.  SERA may remain then a preferred medium for storage and
transfer, it is only more computationally expensive to convert to and from
character addresses during file input-output.\\


\section*{The ISO and the Place of Transliteration}

In the case of Ethiopic a transliteration convention emerged in part from
the absence of a universally recognized character code system to exchange 
electronic text.  In the cases of other writing systems that have had recognized
character code systems for some time -Greek, Cyrillic, Arabic, Hebrew, and Asian
scripts to name a few, electronic communities developed transliteration systems
dispite.\\

Ultimately software, computer systems, and the communications infrastructure
are expected to catch up to the needs of the multilingual community.  Until
then, the need to communicate between people remains too strong a force 
to wait for evolution.  Communities will continue to communicate over
electronic mediums with what ever the lowest common denominator provides.\\

This does not mean chaos reins in the interim.  Rather, members of the virtual
villages are facing this reality head-on and coming to terms with it.  The
International Standards Organization plays host to parties interested in
forming transliteration standards.  Working groups have been formed to set
standards for the transliteration of ten scripts under the ISO technical
subcommittee on the ``Conversion of Written Languages''.  More working groups
will be formed as 1997 progresses.\\

Internationally recognized standards will provide the green light
the software houses need to add support for transliteration systems.  Software
support for a transliteration standard will be vital to the success of the
standard.  \\

At this stage transliteration has reached its primary electronic potential:
the ability to bridge communication in a system of writing between the
software ``haves'' and ``have-nots''.  The paramount point then that the ISO
working groups are well advised to let govern the design of new standards
is that the ultimate success of introduced standards will lie in the finger
tips and keystrokes of the users when they elect to apply the standard when,
returning to the origins,
no specialized software is available to automate the transliteration.
