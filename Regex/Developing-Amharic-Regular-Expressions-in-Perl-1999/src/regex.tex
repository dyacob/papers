
\section*{Developing Amharic Regular Expressions in Perl}
\centerline{\large\textbf{\textit{{\beG}{\daG}{\nG}{\EG}{\lG}{\spaceG}{\yaG}{\IIG}{\qoG}{\bG}}}}
\vspace*{0.4in}


%$plural    = "(\%{\oG}{\cG}(\%{\AG}{\tG})?)";  {\oG}{\cG} {\woG}{\cG} {\AG}{\tG} {\yaG}{\tG} {\AG}{\toG}{\cG} {\yaG}{\toG}{\cG}
%$posseive  = "((\%{\EG}|(?<=\%{\EG}){\yEG}))|{\hG}|{\xG}|{\woG}|(\%{\uG}|(?<=\%{\uG}){\wG}))|(\%11|((?<=\%{\AG}){\waG}))|\%{\AG}{\cG}({\wG}|{\nG}|{\huG}))";

%Simplified

%$plural = "\%{\oG}{\cG}";
%$possesive = "(\%{\EG}|(?<=\%{\EG}){\yEG})";
%$nounrx =  "\A($stem)($plural)?($possessive)?\z";
%$stem = "{\bEG}{\tG}";

\noi
This document has been prepared for members of the \texttt{perl-unicode} email
list to describe by way of example the limitations of working with the Ethiopic
script under the \texttt{UTF-8} support in \texttt{Perl}.  It is the intention
that this document provide the essential information for regular expression 
library developers to add support for syllabic scripts to meet the needs and
expectations of the user community.\\


\noi
Ethiopic is a syllabary whereby, unlike Hebrew and Malayalam, diacritic marks
are \emph{not} used to indicate the vowel element in the \texttt{CV} pair of the
syllable.  This is not to say that Ethiopic syllables are graphically irregular
in the nature of Katakana, Hiragana, or Cherokee, obviously not.  Unlike some
syllabaries diacritic marks in Ethiopic are not graphically detached symbols,
they are not independently encoded.\\

\noi
This indeed is the heart of the problem at hand.  As diacritic marks are not
independently encoded we must extract the vowel (or diphthong) component
of the syllable by analysis of the character code (usually a modulo test).
Such information would be definitive for the syllable's character class.
However programming languages do not yet have support to detect syllabic
classes.  Nor do we have a facility in regex languages to specify the
syllabic class of a character which is essential in linguistic and orthographic
processing.\\

\noi
Lets consider now the Amharic simple plural and first person singular suffixes
as an example of this second problem in particular.  Most any study of
Ethiopian languages applying the Ethiopic syllabary will present possible
formations with these articles in a form similar to:\\

\hspace*{10mm}$<$noun$>$ -o{\cG} -e\\

\noi
\emph{which may also be represented with the \texttt{cv} sequence:}\\

\hspace*{10mm}$<$(cv)(cv)(c)$>$ -vc -v\\

\noi
The morphemes here are -o{\cG} and -e and the sequence is sufficient as a regular
expression for linguists.  Unfortunately the orthography does not follow the
morphology which is the onset to complexity.  There is no way to write in
Ethiopic the vowel element of a syllable which is why IPA symbols tend to be
mixed with Ethiopic elements in language studies.  A consonant in Ethiopic
script on the other hand is generally considered to be the 6$^{th}$ (counting
from 1) syllable (``{\cG}'' in the above for example).\\

\noi
Using the above sequence and the example common noun for ``house'', ``{\bEG}{\tG}'', 
we would obtain the following derivations:\\

\begin{centering}
\begin{tabular}{|ll|}
\hline\hline
  {\bEG}{\tG}        & ``House''       \\
  {\bEG}{\tEG}        & ``My House''    \\
  {\bEG}{\toG}{\cG}      & ``Houses''      \\
  {\bEG}{\toG}{\cEG}      & ``My Houses''   \\
\hline\hline
\end{tabular}\\
\end{centering}

~\\

\noi
This is indeed a very simple example, as there are 13 possessive forms that can
follow the base noun, 9 of which can also follow the 6 pluralizing articles of 
Amharic common nouns.  Valid combinations of the full collection of Amharic
prefixes and suffixes can lead to derivations in the thousands.  Fortunately
it is all very regular and predictable. Only the logical expression in native
script in a form following our linguistic understanding proves difficult.

\subsection*{Transliteration}

\noi
Now we will build a regular expression for our example using Roman script.
Transliteration into Roman script provides a work-around to the 
phoneme-morpheme boundary problem where the \texttt{cv} pairs of the
syllables are split up and discretely known.  We define then
our regular expression in Perl with the aid of a few useful strings:

\begin{alltt}
  $plural = "oc";       # aka  -o{\cG}
  $pos    = "e";        # aka  -e
  $stem   = "bet";

  /\(\backslash\)A$stem($plural)?($pos)?\(\backslash\)z/
\end{alltt}

%\begin{alltt}
%  /\(\backslash\)Abet($plural)?($pos)?\(\backslash\)z/
%  /\(\backslash\)Abet(oc)?(e)?\(\backslash\)z/
%\end{alltt}

\noi
Our regular expression expands into the expected valid formations:\\

\noi
\begin{tabular}{|lclcll|}
\hline\hline
  \dotable{Transliterated}{Regex}
             &               & \dotable{Resultant}{~} 
                                      &               & \dotable{Retransliterated}{Resultant}
                                                               & \dotable{English}{Meaning} \\
\hline
  bet        & $\Rightarrow$ & bet    & $\Rightarrow$ & {\bEG}{\tG}   & ``House''    \\
  bet(oc)    & $\Rightarrow$ & betoc  & $\Rightarrow$ & {\bEG}{\toG}{\cG} & ``Houses''   \\
  bet(e)     & $\Rightarrow$ & bete   & $\Rightarrow$ & {\bEG}{\tEG}   & ``My House'' \\
  bet(oc)(e) & $\Rightarrow$ & betoce & $\Rightarrow$ & {\bEG}{\toG}{\cEG} & ``My Houses'' \\
\hline\hline
\end{tabular}\\
~\\

\noi
The primary advantage to using transliteration in regular expressions is that
we never have to be concerned with handling the recombination problem of the
consonant and vowel across the subexpression boundaries\footnote{Across all
grouping, character class and alternation boundaries.}.
What we loose primarily then is the practicality of working in 
native script and the economic cost, in computing terms, of the transliteration
and retransliteration work.


\subsection*{Faking Character Classes}

\noi
Given the \texttt{UTF-8} support in newer versions of \texttt{Perl} we would
hope to be able to be to leave transliteration behind.  The Ethiopic programmer
would most certainly be eager to do so and attempting to construct a regular
expression for our four derivations would be lead most directly to:

\begin{alltt}
  /\(\backslash\)A{\bEG}([{\tG}{\tEG}]|({\toG}[{\cG}{\cEG}]))\(\backslash\)z/
\end{alltt}

\noi
Unfortunately it was required that we \texttt{chop} off the ``{\tG}'' in our
\texttt{\$stem} ``{\bEG}{\tG}'' and build it into the groupings and character classes
of our regular expression.  We have lost the convenience of our
\texttt{\$plural} and \texttt{\$pos} abstractions and had to expand them into
our subexpressions as well.\\

\noi
This is of course all overly specific to our example word ``{\bEG}{\tG}'' or at least
could be not be readily applied to words ending in other than ``{\tG}''.  
As suggested in the 1997 paper\footnote{``Thoughts on Regular Expressions for Ethiopic''
\texttt{ftp://ftp.cs.indiana.edu/pub/fidel/perl-unicode/regex979.ps.gz}}
macros could be defined to help generalize our expressions:\\

\noi
%\fbox{
\begin{tabular}{>{\tt}r >{\tt}c >{\tt}l}
  \$HAMS &=& [{\hEG}{\lEG}{\HEG}{\mEG}{\ssEG}{\rEG}{\sEG}{\xEG}{\qEG}{\QEG}{\bEG}{\vEG}{\tEG}{\cEG}{\hhEG}{\nEG}{\NEG}{\EG}{\kEG}{\KEG}{\wEG}{\EEG}{\zEG}{\ZEG}{\yEG}{\dEG}{\DEG}{\jEG}{\gEG}{\GEG}{\TEG}{\CEG}{\PEG}{\SEG}{\SSEG}{\fEG}{\pEG}];\\
  \$SADS &=& [{\hG}{\lG}{\HG}{\mG}{\ssG}{\rG}{\sG}{\xG}{\qG}{\QG}{\bG}{\vG}{\tG}{\cG}{\hG}{\nG}{\NG}{\IG}{\kG}{\KG}{\wG}{\zG}{\ZG}{\yG}{\dG}{\DG}{\jG}{\gG}{\GG}{\TG}{\CG}{\PG}{\SG}{\SSG}{\fG}{\pG}];\\
  \$SABI &=& [{\hoG}{\loG}{\HoG}{\moG}{\ssoG}{\roG}{\soG}{\xoG}{\qoG}{\QoG}{\boG}{\voG}{\toG}{\coG}{\hhoG}{\noG}{\NoG}{\oG}{\koG}{\KoG}{\woG}{\ooG}{\zoG}{\ZoG}{\yoG}{\doG}{\DoG}{\joG}{\goG}{\GoG}{\ToG}{\CoG}{\PoG}{\SoG}{\SSoG}{\foG}{\poG}];\\
\end{tabular}\\ %}\\

\noi
For an arbitrary stem we may then construct:
 
\begin{alltt}
  my ($lastChar) = chop ($stem);
  if ( /\(\backslash\)A$stem($SADS|$HAMS|($SABI[{\cG}{\cEG}]))\(\backslash\)z/ ) \{
    \vdots
  \}
  $stem .= $lastChar;
\end{alltt}

\noi
Test of the form:
\begin{alltt}
  if ( $word =~ /($SADS|$HAMS|($SABI[{\cG}{\cEG}]))\(\backslash\)z/ ) \{ \ldots
\end{alltt}

\noi
are a bit better suited.

\subsection*{Defining Named Properties}

\noi
We may still find the above approach to be inefficient and increasingly
memory consuming and expressively ungainly as we would ultimately extend
our regular expression beyond our simple singular instances of our example
suffixes.\\

\noi
With recent development versions of \texttt{Perl}\footnote{\texttt{Perl} versions \texttt{5.005\_52} and later.} we are given the capability to define new character properties
through the newly introduced \texttt{$\backslash$p} escape.  The author has
done exactly this in the package found at:\\

\texttt{ftp://ftp.cs.indiana.edu/pub/fidel/perl-unicode/perl-unicode.tar.gz}.\\

\noi
The package adds the character property definitions for Ethiopic that at this
time have not made it into the pre-Unicode 3.0 based \emph{UnicodeData} file.
The package adds the new property type $Y_{i}$ to identify the 
s\underline{Y}llabic order $i$ of a syllable.\\
 
\noi
Ultimately however without the ability to check the character property values
of the terminal characters in preceding subexpressions we return to the
same expressive limitations we encountered with the faked character classes:

\begin{alltt}
  if ( $word =~ /[\(\backslash\)p\{Y5\}\(\backslash\)p\{Y4\}]|(\(\backslash\)p\{Y6\}[{\cG}{\cEG}]))\(\backslash\)z/ ) \{ \ldots
\end{alltt}

%\noi
%problem is loss of generality.
%we could not insert a stem into a regex of the above :\\

%\begin{alltt}
%  $prefix = "[{\beG}{\keG}{\yeG}]";
%  $suffix = "(\(\backslash\)p\{Y5\}|\(\backslash\)p\{Y4\}|(\(\backslash\)p\{Y6\}({\cG}|{\cEG})))";

%  \(\backslash\)A($prefix)?$stem($suffix)\(\backslash\)z/

%  $pos     = "\(\backslash\)p\{Y4\}";
%  $plural  = "\(\backslash\)p\{Y6\}{\cG}";

%  $word   =~ /\(\backslash\)A($prefix)?(\(\backslash\)w\{2,5\})(($plural)?($pos))?\(\backslash\)z/;
%  $stem   =  $2;
 
%  $pos    =  "(?<=\(\backslash\)p\{Y4\})";   \\
%  $plural =  "(?<=\(\backslash\)p\{Y6\}){\cG}"; \\
%\end{alltt} 

%\noi
%This works a little better


\subsection*{Back to the \texttt{\%} Operator}
\noi
In the 1997 paper the \texttt{\%} operator was proposed for matching the
syllabic class (vowel or diphthong property) of any preceding adjacent
character that has a syllabic context\footnote{Essentially \texttt{\%}
would work like \texttt{(?<=$\backslash$p\{Y$_i$\})} across preceeding
subexpressions.}.  The syllabary operator still seems to offer a certain
convenience:

%\begin{tabular}{>{\tt}l>{\tt}c>{\tt}l>{\tt}l}
\begin{alltt}
  $plural = "%6{\cG}";   # aka  -o{\cG}
  $pos    = "%4";     # aka  -e
   
  $stem   = "{\bEG}{\tG}";    # we stick with our example though the $stem
                      # can now be arbitrary

  /\(\backslash\)A$stem($plural)?($pos)?\(\backslash\)z/  # Same as the transliterated regex!
\end{alltt}
%  \(\backslash\)A{\bEG}{\tG}($plural)?($pos)?\(\backslash\)z
%  \(\backslash\)A{\bEG}{\tG}(%6{\cG})?(%4)?\(\backslash\)z


\noi
Our regular expression expands into the expected valid formations:\\

\noi
\begin{tabular}{|lclcll|}
\hline\hline
  Regex           &               & Expansion
                  &               & Resultant  & English Meaning \\
\hline
  {\bEG}{\tG}            & $\Rightarrow$ & \emph{none}           
                  & $\Rightarrow$ & {\bEG}{\tG}       & ``House''     \\
  {\bEG}{\tG}(\%6{\cG})     & $\Rightarrow$ & {\bEG}({\tG}\%6){\cG}         
                  & $\Rightarrow$ & {\bEG}{\toG}{\cG}     & ``Houses''    \\
  {\bEG}{\tG}(\%4)       & $\Rightarrow$ & {\bEG}({\tG}\%4)       
                  & $\Rightarrow$ & {\bEG}{\tEG}       & ``My House''  \\
  {\bEG}{\tG}(\%6{\cG})(\%4)& $\Rightarrow$ & {\bEG}({\tG}\%6)({\cG}\%4) 
                  & $\Rightarrow$ & {\bEG}{\toG}{\cEG}     & ``My Houses'' \\ 
\hline\hline
\end{tabular}\\
~\\

\noi
The above expansions demonstrates what is truly desired in this field of text
processing.  The capability to compose general vs specific (expanded
subexpressions) regular expressions in native script is also fundamental 
to being able to utilize the regular expressions language.
Another generalized expression:

\begin{alltt}
  $prefix = "[{\beG}{\keG}{\yeG}]";
  $plural = "%6{\cG}";        # aka  -o{\cG}
  $pos    = "%4";          # aka  -e

  if ( $word   =~ /\(\backslash\)A($prefix)?(\(\backslash\)w\{2,5\})(($plural)?($pos))?\(\backslash\)z/ ) \{
    $stem =  $2;
      \vdots
  \}
\end{alltt}

\noi
The use of the syllabary class operator offers the convenience of constructing
a regular expression in a form not far removed from how the orthographic
morphology is understood.  The apparent necessity of the operator would
diminish were an escape construct capable of performing the same service.
The author invites comments and suggestions to this end.









